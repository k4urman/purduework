\documentclass[11pt]{article}
\usepackage{comment}
\usepackage{graphicx}
\usepackage{fourier-orns}
\usepackage{algorithm, algpseudocode} % For pseudo code
\usepackage{tabto}
\usepackage{amsmath, amsthm, amssymb, amsfonts} % Linear algebra and logic related symbols
\usepackage{color}
\usepackage{fancyhdr} % Header
\usepackage{enumitem} % More option for enumeration and itemization
%\usepackage{palatino} % font
\usepackage{latexsym}
\usepackage{multicol}
\usepackage{fourier-orns}
\usepackage{algorithm, algpseudocode} % For pseudo code
\usepackage{tabto}
\usepackage{amsmath, amsthm, amssymb, amsfonts}
\usepackage{color}
\usepackage{fancyhdr} % Header
\usepackage{enumitem} % More option for enumeration and itemization
\usepackage{tikz}
\usetikzlibrary{arrows} % graph
\usepackage{url}
\topmargin=-.5in    %0cm
\textheight=9.0in     %24.1cm            
\evensidemargin=0in %0cm
\oddsidemargin=0.0in  %-.3cm
\textwidth=6.5in    %16.5cm 

\title{CS182 Homework \# 2}
\author{Insert}

\begin{document}

\maketitle

\section*{Question 0}
\begin{quote}
    \textit{I, Maninder (Kaurman) Kaur, affirm that I have not given or received any unauthorized help on this assignment and that this work is my own. What I have submitted is expressed and explained in my own words. I have not used any online websites that provide a solution. I will not post any parts of this problem set to any online platform and doing so is a violation of course policy.}
\end{quote}

\clearpage
\section*{Question 1}

    \textbf{Let A, B be the sets \\
A= \(\{3,8,\varnothing\}\) \\
B = \(\{1, 3, \{2, 3\}, 4, 5, 8, 9\}\) \\
C = \(\{B, 8, \{0, 3, 9\}, 9\}\) \\
Answer the following questions.}

    \subsection*{Solution}
    \begin{enumerate}[label=(\alph*)]
        \item \textbf{What is \(|B|\)} The absolute value bars indicates cardinality, or how many elements are in the set, so 8.\\
        \item \textbf{What is \(P(A)\)} The P indicates power set, or all possible sets that can be created by the set. We can do this by using the formula \(2^n\) where n is the cardinality. \(2^3\) is 8.\\
        \item \textbf{What is \(A-B\)} The - means what values are in the first set, that are not in the second. All sets have \(\varnothing\) in them by default, but A has an extra, so \(\varnothing.\)\\
        \item \textbf{What is \(A \cap B \cap C\)} The \(\cap\) means intersection or all common elements which are \{3, 8, \(\varnothing\}\) \\
        \item \textbf{What is \((A \cup B) \cap C\)} The \(\cap\) creates a bigger set that includes all elements without repeat. The we intersect with C, ending up with \(\{1, 3, \{2, 3\}, 4, 5, 8, 9\}\). \\
        \item \textbf{What is \(A *\{2,4\}\)} The x refers to Cartesian Product, which is \(\{\{3,2\},\{3,4\},\{8,2\},\{8,4\},\{\varnothing,2\},\{\varnothing,4\}\}\) \\
        \item \textbf{T/F \(0 \in C\)} is T, because 0 is indeed apart of C.\\ 
        \item \textbf{T/F \(\varnothing \subseteq C\)} is T, because the empty set is always a subset of a set. \\
        \item \textbf{T/F \(\{2,3\} \subset A \cup B\)} means is \(\{2,3\}\) a proper subset of \(A \cup B\), and it is T, because it is apart of the two, but does not equal them. \\
        \item \textbf{T/F \((A-B) - \varnothing = \varnothing\)} From (c), we can see that the answer is an empty set. An empty set - a empty set will result in a empty set, so T/\\
        \item \textbf{T/F \(|P(A * C)| = 2048\)} The powerset is \(2^12\), from the cartesian product, which is not 2048, so F. \\
        \item \textbf{T/F \(\{8,9\} \in P(C)\)} is T, because a powerset includes all possible sets from C, and one of those sets must be \(\{8,9\}\) \\

    \end{enumerate}

\clearpage
\section*{Question 2}
    \textbf{Let the function \(f: \mathbb{R} \rightarrow \mathbb{R} be defined by\) \\
    \(f(x) = \begin{cases}
        x & \text{if } x <0 \\
        \frac{x^2}{2} & \text{if } 0 \leq x \leq 2 \\
        \frac{1}{x-2}+1 & \text{if } x> 2
    \end{cases}\) \\ \\
    Prove of disprove the following:}
    \subsection*{Solution}
    \begin{enumerate}[label=(\alph*)]
        \item \textbf{\(f\) is injective} To disprove one-to-one, where no two x can have the same y, use x=2 or 3, because both share y=2.
        \item \textbf{\(f\) is surjective} is proved through the if statements in the peicewise. All values of x that are REAL have a value, even if they share on, and it goes to infinity in both directions.
        \item \textbf{\(f\) is bijective} Because injective was disproved, by default it is not bijective as the function would have to be both one-to-one and onto. 
    \end{enumerate}

\end{document}
