\documentclass[11pt]{article}
\usepackage{comment}
\usepackage{graphicx}
\usepackage{fourier-orns}
\usepackage{algorithm, algpseudocode} % For pseudo code
\usepackage{tabto}
\usepackage{amsmath, amsthm, amssymb, amsfonts} % Linear algebra and logic related symbols
\usepackage{color}
\usepackage{fancyhdr} % Header
\usepackage{enumitem} % More option for enumeration and itemization
%\usepackage{palatino} % font
\usepackage{latexsym}
\usepackage{multicol}
\usepackage{fourier-orns}
\usepackage{algorithm, algpseudocode} % For pseudo code
\usepackage{tabto}
\usepackage{amsmath, amsthm, amssymb, amsfonts}
\usepackage{color}
\usepackage{fancyhdr} % Header
\usepackage{enumitem} % More option for enumeration and itemization
\usepackage{tikz}
\usetikzlibrary{arrows} % graph
\usepackage{url}
\topmargin=-.5in    %0cm
\textheight=9.0in     %24.1cm            
\evensidemargin=0in %0cm
\oddsidemargin=0.0in  %-.3cm
\textwidth=6.5in    %16.5cm 

\title{CS182 Homework \# 2}
\author{Insert}

\begin{document}

\maketitle

\section*{Question 0}
\begin{quote}
    \textit{I, Maninder (Kaurman) Kaur, affirm that I have not given or received any unauthorized help on this assignment and that this work is my own. What I have submitted is expressed and explained in my own words. I have not used any online websites that provide a solution. I will not post any parts of this problem set to any online platform and doing so is a violation of course policy.}
\end{quote}

\clearpage
\section*{Question 1}

    \textbf{Let A, B be the sets \\
A= \(\{3,8,\varnothing\}\) \\
B = \(\{1, 3, \{2, 3\}, 4, 5, 8, 9\}\) \\
C = \(\{B, 8, \{0, 3, 9\}, 9\}\) \\
Answer the following questions.}

    \subsection*{Solution}
    \begin{enumerate}[label=(\alph*)]
        \item \textbf{What is \(|B|\)} The absolute value bars indicates cardinality, or how many elements are in the set, so 8.\\
        \item \textbf{What is \(P(A)\)} The P indicates power set, or all possible sets that can be created by the set. We can do this by using the formula \(2^n\) where n is the cardinality. \(2^3\) is 8.\\
        \item \textbf{What is \(A-B\)} The - means what values are in the first set, that are not in the second. All sets have \(\varnothing\) in them by default, but A has an extra, so \(\varnothing.\)\\
        \item \textbf{What is \(A \cap B \cap C\)} The \(\cap\) means intersection or all common elements which are \{3, 8, \(\varnothing\}\) \\
        \item \textbf{What is \((A \cup B) \cap C\)} The \(\cap\) creates a bigger set that includes all elements without repeat. The we intersect with C, ending up with \(\{1, 3, \{2, 3\}, 4, 5, 8, 9\}\). \\
        \item \textbf{What is \(A *\{2,4\}\)} The x refers to Cartesian Product, which is \(\{\{3,2\},\{3,4\},\{8,2\},\{8,4\},\{\varnothing,2\},\{\varnothing,4\}\}\) \\
        \item \textbf{T/F \(0 \in C\)} is T, because 0 is indeed apart of C.\\ 
        \item \textbf{T/F \(\varnothing \subseteq C\)} is T, because the empty set is always a subset of a set. \\
        \item \textbf{T/F \(\{2,3\} \subset A \cup B\)} means is \(\{2,3\}\) a proper subset of \(A \cup B\), and it is T, because it is apart of the two, but does not equal them. \\
        \item \textbf{T/F \((A-B) - \varnothing = \varnothing\)} From (c), we can see that the answer is an empty set. An empty set - a empty set will result in a empty set, so T/\\
        \item \textbf{T/F \(|P(A * C)| = 2048\)} The powerset is \(2^12\), from the cartesian product, which is not 2048, so F. \\
        \item \textbf{T/F \(\{8,9\} \in P(C)\)} is T, because a powerset includes all possible sets from C, and one of those sets must be \(\{8,9\}\) \\

    \end{enumerate}

\clearpage
\section*{Question 2}
    \textbf{Let the function \(f: \mathbb{R} \rightarrow \mathbb{R} be defined by\) \\
    \(f(x) = \begin{cases}
        x & \text{if } x <0 \\
        \frac{x^2}{2} & \text{if } 0 \leq x \leq 2 \\
        \frac{1}{x-2}+1 & \text{if } x> 2
    \end{cases}\) \\ \\
    Prove of disprove the following:}
    \subsection*{Solution}
    \begin{enumerate}[label=(\alph*)]
        \item \textbf{\(f\) is injective} To disprove one-to-one, where no two x can have the same y, use x=2 or 3, because both share y=2.
        \item \textbf{\(f\) is surjective} is proved through the if statements in the peicewise. All values of x that are REAL have a value, even if they share on, and it goes to infinity in both directions.
        \item \textbf{\(f\) is bijective} Because injective was disproved, by default it is not bijective as the function would have to be both one-to-one and onto. 
    \end{enumerate}

\clearpage
\section*{Question 3}
    \textbf{Prove that there does not exist a least negative rational number. For example, -1 is less negative than -2.5 which is less negative than -3.3.}
    \subsection*{Solution}
    \begin{enumerate}[label=(\alph*)]
        \item[] Lets say there is a least negative rational number that we have named \(x\) (just go along with it). This means for any other negative rational number, \(n\), \(x \leq n\). \\ \\
        All rational numbers can be re-written as fractions like so: \(x = \frac{p}{q}\), where p and q are integers and q is NOT 0.\\ \\
        \(y=\frac{x}{2} \rightarrow \frac{p/q}{2} = \frac{p}{2q} \) \\ \\
        The equation makes \(y < x\). We can repeat this process with y and so on, and each time it will continue to become smaller and smaller.
        
    \end{enumerate}

\clearpage
\section*{Question 4}
    \textbf{Prove that if \(x^2 +7\) is odd, then \(x\) is even, \(\forall x \in \mathbb{Z}\)}
    \subsection*{Solution}
    \begin{enumerate}[label=(\alph*)]
        \item[] Let \(n = 2k\) if it is even \\
        Let \(n= 2k+1\) \(\forall k \in \mathbb{Z}\) if it is odd. We will use this one to prove by contrapositive that "if x is odd, the equation will be even". 
        Replace x with the above odd equation like so: \\ \\\(x^2 + 7 \rightarrow (2k+1)^2 + 7 \rightarrow\) \\
        \((2k+1)(2k+1) + 7 \rightarrow 4k^2+4k+8 \rightarrow\) \\ \\
        \(2(2k^2 + 2k+4)\) Let the values in the parentheses be some integer, which no resembles 2k, just like \(n=2k\), which means it is even! We have proved the contrapositive.
    \end{enumerate}

\clearpage
\section*{Question 5}
    \textbf{Prove that \(5x^2 +4y^2+20y+22=73\) has no integer solutions. [Hint: Try narrowing down the range of x or y (proof by cases) by completing the square.
Then, use proof by contradiction: suppose an integer solution exists.]}
    \subsection*{Solution}
    \begin{enumerate}[label=(\alph*)]
        \item[] \(5x^2 +4y^2+20y+22=73\) \\
        \(5x^2 +4(y^2+5y)+22=73\) \\
        \(5x^2+4(y^2+5y(\frac{5}{2})^2-(\frac{5}{2})^2)\) +22 =73 \\
        \(5x^2+4(y^2+5y(\frac{5}{2})^2-\frac{25}{4})\) +22=73 \\
        \(5x^2+4(y+\frac{5}{2})-3=73\) \\
        \(5x^2+4(y+\frac{5}{2})=76\) \\
        \(4(y+\frac{5}{2})^2=76-5x^2\) \\
        \textbf{PROOF}: Proof by contradiction, there exists such integer solution. \\
        \textbf{POSSIBLE INPUTS:} \(0,\pm 1,\pm 2,\pm 3\) \\
        \(x = 0: 76 - 5 \cdot 0 = 76, \frac{76}{4} = 19\) \\
        \(x = \pm 1: 76 - 5 \cdot 1 = 71, \frac{71}{4} = 17.75\) \\
        \(x = \pm 2: 76 - 5 \cdot 4 = 56, \frac{56}{4} = 14\) \\
        \(x = \pm 3: 76 - 5 \cdot 9 = 76 - 45 = 31, \frac{31}{4} = 7.75\) \\
        Once we plug them in, we find out that none of the inputs for x provide a perfect square or integer output, meaning that \(5x^2 +4y^2+20y+22=73\) indeed has no integer solutions. \\
        
        
    \end{enumerate}

\end{document}
