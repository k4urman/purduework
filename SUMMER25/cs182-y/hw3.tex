\documentclass[11pt]{article}
\usepackage{comment}
\usepackage{graphicx}
\usepackage{fourier-orns}
\usepackage{algorithm, algpseudocode} % For pseudo code
\usepackage{tabto}
\usepackage{amsmath, amsthm, amssymb, amsfonts} % Linear algebra and logic related symbols
\usepackage{color}
\usepackage{fancyhdr} % Header
\usepackage{enumitem} % More option for enumeration and itemization
%\usepackage{palatino} % font
\usepackage{latexsym}
\usepackage{multicol}
\usepackage{fourier-orns}
\usepackage{algorithm, algpseudocode} % For pseudo code
\usepackage{tabto}
\usepackage{amsmath, amsthm, amssymb, amsfonts}
\usepackage{color}
\usepackage{fancyhdr} % Header
\usepackage{enumitem} % More option for enumeration and itemization
\usepackage{tikz}
\usetikzlibrary{arrows} % graph
\usepackage{url}
\topmargin=-.5in    %0cm
\textheight=9.0in     %24.1cm            
\evensidemargin=0in %0cm
\oddsidemargin=0.0in  %-.3cm
\textwidth=6.5in    %16.5cm 

\title{CS182 Homework \# 3}
\author{Maninder (Kaurman) Kaur}

\begin{document}

\maketitle

\section*{Question 0}
\begin{quote}
    \textit{I, Maninder (Kaurman) Kaur, affirm that I have not given or received any unauthorized help on this assignment and that this work is my own. What I have submitted is expressed and explained in my own words. I have not used any online websites that provide a solution. I will not post any parts of this problem set to any online platform and doing so is a violation of course policy.}
\end{quote}

\clearpage
\section*{Question 1}

    \textbf{Problem 1. (4 $\times$ 4 points) Prove or disprove that the relations given below are symmetric, reflexive, and transitive. Hence, show whether or not the relations are equivalence relations.}
    \begin{enumerate}[label=(\alph*)]
        \item \textbf{Let $A = \{ \text{all Purdue students} \}$. $R = \{ (a, b) \in A \times A \mid a \text{ takes a class that } b \text{ does not take} \}$}
        \item \textbf{Let $S = \{ \text{Rock, Paper, Scissors} \}$. $R = \{ (x, y) \in S \times S \mid x \text{ beats } y \text{ in the game of Rock-Paper-Scissors } \}$}
        \item \textbf{$R = \{ (x, y) \in \mathbb{R} \times \mathbb{R} \mid |x - y| \leq \min \{|x|, |y|\} \}$}
        \item \textbf{Let $L$ be the set of lines in a plane. $R = \{ (x, y) \in L \times L \mid x \text{ is parallel to } y \}$}
    \end{enumerate}

    \subsection*{Solution}
    \begin{enumerate}[label=(\alph*)]
        \item 
    \end{enumerate}


\clearpage
\section*{Question 2}

    \textbf{Problem 2. (4 $\times$ 3 points) Design a relation for each of the following criteria and explain briefly.}
    \begin{enumerate}[label=(\alph*)]
        \item \textbf{Determine the minimum number of pairs we would have to add to $R$ to obtain a reflexive relation $R_1$, and list those pairs.}
        \item \textbf{Determine the minimum number of pairs we would have to add to $R$ to obtain a reflexive and symmetric relation $R_2$, and list those pairs.}
        \item \textbf{Determine the minimum number of pairs we would have to add to $R$ to obtain a reflexive, symmetric, and transitive relation $R_{eq}$, and list those pairs.}
    \end{enumerate}

    \subsection*{Solution}
    \begin{enumerate}[label=(\alph*)]
        \item
    \end{enumerate}


\clearpage
\section*{Question 3}

    \textbf{Problem 3. (7 points) Let $U = \{a, b, c\}$, and let $S$ be the power set of $U$, $P(U)$. Define a relation $R$ on $S$ as follows: $R = \{ (A, B) \in P(U) \times P(U) \mid A \cap B \neq \emptyset \}$. Prove or disprove that this relation is reflexive, symmetric, and transitive. Hence, show whether or not the relation is an equivalence relation.}

    \subsection*{Solution}
    \begin{enumerate}[label=(\alph*)]
        \item Write solution here $x=a$
    \end{enumerate}


\clearpage
\section*{Question 4}

    \textbf{Problem 4. (10 points) Let $A = \{1, 2, 3, 4\}$ and $R$ be a relation on $A$ given by $R = \{(2, 1), (1, 4), (3, 3)\}$. Observe that $R$ is neither reflexive, nor symmetric, nor transitive. Sometimes we may want to extend a relation $R$ into an equivalence relation (one that is reflexive, symmetric, and transitive). For instance, $A$ could represent train stations in a railway network, and pairs in $R$ like $(2, 1)$ and $(1, 4)$ could be distinct non-stop train journeys. If we extend $R$ by the least number of pairs necessary into an equivalence relation $R_{eq}$, then we obtain the set of pairs $(a, b)$ where $b$ is reachable from $a$ by way of zero or more trains. It is key that we extend $R$ by the \textit{fewest pairs possible}. Indeed, while $A \times A$ contains $R$, it also includes many unrelated pairs, so it is not the minimal extension we seek. Solve the following questions:}
    \begin{enumerate}[label=(\alph*)]
        \item \textbf{Determine the minimum number of pairs we would have to add to $R$ to obtain a reflexive relation $R_1$, and list those pairs.}
        \item \textbf{Determine the minimum number of pairs we would have to add to $R$ to obtain a reflexive and symmetric relation $R_2$, and list those pairs.}
        \item \textbf{Determine the minimum number of pairs we would have to add to $R$ to obtain a reflexive, symmetric, and transitive relation $R_{eq}$, and list those pairs.}
    \end{enumerate}


    \subsection*{Solution}
    \begin{enumerate}[label=(\alph*)]
        \item 
    \end{enumerate}


\clearpage
\section*{Question 5}

    \textbf{Problem 5. (4 $\times$ 3 points) Solve the following questions: Show your work to receive full credit.}
    \begin{enumerate}[label=(\alph*)]
        \item \textbf{Compute $7^{614} \pmod{8}$}
        \item \textbf{Compute $(3^{196} + 5^{88}) \pmod{77}$}
        \item \textbf{Compute $(123 \cdot 456) \pmod{78}$}
    \end{enumerate}


    \subsection*{Solution}
    \begin{enumerate}[label=(\alph*)]
        \item
    \end{enumerate}


\clearpage
\section*{Question 6}

    \textbf{Problem 6. (3 $\times$ 2 + 4 points) Solve the following questions:}
    \begin{enumerate}[label=(\alph*)]
        \item \textbf{Calculate the GCD of 120 and 45 using Euclidean Algorithm}
        \item \textbf{Calculate the GCD of 246 and 230 using Euclidean Algorithm}
        \item \textbf{Bézout’s Theorem is that if $a$ and $b$ are positive integers, then there exist integers $s$ and $t$ such that $\text{gcd}(a, b) = sa + tb$. Using the extended Euclidean algorithm, find integers $s$ and $t$ such that $99s + 78t = \text{gcd}(99, 78)$.}
    \end{enumerate}

    \subsection*{Solution}
    \begin{enumerate}[label=(\alph*)]
        \item 
    \end{enumerate}


\clearpage
\section*{Question 7}

    \textbf{Problem 7. (6 + 5 points) Hash functions can also be used for strings. One example is the polynomial rolling hash function. Our polynomial rolling hash function $h(s)$ is defined as follow:
    \( h(s) = \left( \sum_{i=0}^{n-1} \text{val}(s[i]) \cdot p^i \right) \pmod{m} \)
    where $n$ is the number of letters in the string $s$, \(\text{val}(s[i])\) is an integer value corresponding to the $i$ th letter of the string $s$ (A=1, B = 2, ..., Z = 26), $p$ is a prime constant $5$, and $m$ is the size of the table, let’s say 11. We want to insert [CAT, DOG, BIRD, PONY] in this specific order. Solve the following questions:}

    \subsection*{Solution}
    \begin{enumerate}[label=(\alph*)]
        \item Calculate the initial hash index for each of the given keys without considering collisions. Are there any collisions?
        \item Collisions are a common issue because we want to balance the memory usage with other considerations. There are different methods to deal with collisions. In this problem, we will explore one method called \textit{open addressing} in which we handle collisions by \textit{probing} (searching) through a sequence of alternative locations until we find an open spot. Consider the probing function $p(s, i) = (h(s) + 3i) \pmod{m}$ where we initially check $p(s, 0) = h(s)$ and then jump forward three slots at a time to try to find an open slot. In other words, if some other string is already stored in $p(s, 0) = h(s)$, then we check $p(s, 1) = (h(s) + 3) \pmod{m}$ and if that one is already full, we check $p(s, 2) = (h(s) + 6) \pmod{m}$ and so on until we find an open spot. The advantage of open addressing is that we can handle collisions without changing our approach of only storing one string per slot in the hash table. The disadvantage is that searching for a string is more complicated because we also have to check the alternative locations in order until we find that string or an open slot. Given $p(s, i) = (h(s) + 3i) \pmod{m}$, redo part (a) and compute the hash index for the same sequence of strings taking into account collisions and the probing function $p(s, i)$.
    \end{enumerate}


\clearpage
\section*{Question 8}

    \textbf{Problem 8. (20 points) In Minecraft, a seed is a number or even a word that determines the way the world is generated. The seeds are assigned using a Pseudorandom Number Generator(PRNG) to create a unique and seemingly random world for players to explore. You luckily got a world with beautiful cherry blossoms and you want to find the seed that creates a world where cherry blossoms are guaranteed to appear at spawn. Instead of randomly searching for seeds, which takes a long time, your goal is to find them by applying reverse engineering to the game’s code. Our goal is to work backwards with the condition for cherry groves to find the seed we want. The algorithm for placing a large Cherry Grove Biome at the world’s origin depends on the third output, $x_3$, of the server’s custom Linear Congruential Generator(LCG). The generation algorithm is as follows:
    \( x_{n+1} = (ax_n + c) \pmod{m} \)
    where $m = 2^6$, $a = 5^{18}$, $c = 8$. Solve the following questions:}

    \subsection*{Solution}
    \begin{enumerate}[label=(\alph*)]
        \item If the initial seed $x_0$ is 50, what is the next 2 numbers generated by the LCG?
        \item To reverse the process, you must be able to find the previous number in the sequence, which requires finding the modular multiplicative inverse of $a$. The modular multiplicative inverse is an integer $a^{-1}$ such that: $aa^{-1} \equiv 1 \pmod{m}$. Find the smallest positive $a^{-1}$.
        \item Make a formula for $x_n$ in terms of $x_{n+1}$ using $a^{-1}$ you found in (b) [hint: simplify the equation first]
        \item To guarantee the cherry blossoms at spawn, the third output of LCG, $x_3$, should be exactly 10. Using the reverse formula you found in part (b), find the initial seed $x_0$. Note: The numbers in this question are completely irrelevant to the actual Minecraft system.
    \end{enumerate}


\end{document}
