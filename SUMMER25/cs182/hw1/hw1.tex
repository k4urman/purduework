\documentclass[11pt]{article}
\usepackage{comment}
\usepackage{graphicx}
\usepackage{fourier-orns}
\usepackage{algorithm, algpseudocode} % For pseudo code
\usepackage{tabto}
\usepackage{amsmath, amsthm, amssymb, amsfonts} % Linear algebra and logic related symbols
\usepackage{color}
\usepackage{fancyhdr} % Header
\usepackage{enumitem} % More option for enumeration and itemization
%\usepackage{palatino} % font
\usepackage{latexsym}
\usepackage{multicol}
\usepackage{fourier-orns}
\usepackage{algorithm, algpseudocode} % For pseudo code
\usepackage{tabto}
\usepackage{amsmath, amsthm, amssymb, amsfonts}
\usepackage{color}
\usepackage{fancyhdr} % Header
\usepackage{enumitem} % More option for enumeration and itemization
\usepackage{tikz}
\usetikzlibrary{arrows} % graph
\usepackage{url}
\topmargin=-.5in    %0cm
\textheight=9.0in     %24.1cm            
\evensidemargin=0in %0cm
\oddsidemargin=0.0in  %-.3cm
\textwidth=6.5in    %16.5cm 

\title{CS182 Homework \# 1}
\author{Maninder (Kaurman) Kaur}

\begin{document}

\maketitle

\section*{Question 0}
\begin{quote}
    \textit{I, Maninder (Kaurman) Kaur, affirm that I have not given or received any unauthorized help on this assignment and that this work is my own. What I have submitted is expressed and explained in my own words. I have not used any online websites that provide a solution. I will not post any parts of this problem set to any online platform and doing so is a violation of course policy.}
\end{quote}

\clearpage
\section*{Question 1}

    \textbf{Let P(x) be the statement "\(x^2 \geq x.\)" Suppose the domain consists of real numbers (i.e., \(x \in R\)). Determine the truth values of the following:}

    \begin{enumerate}[label=(\alph*)]
        \item \(P(2)\)
        \item \(P(0.5)\)
        \item \(P(-3)\)
        \item \(\exists x \neg P(x)\)
        \item \(\exists x P(x)\)
        \item \(\forall x P(x)\)
    \end{enumerate}
    \subsection*{Solution}
    \begin{enumerate}[label=(\alph*)]
        \item Plug in the value of x into the equation like so: \(P(2) = 2^2 \geq 2 \equiv 4 \geq 2\) which is TRUE.
        \item Plug in the value of x into the equation like so: \(P(.5) = .5^2 \geq .5 \equiv .25 \geq .5\) which is FALSE.
        \item Plug in the value of x into the equation like so: \(P(-3) = (-3)^2 \geq -3 \equiv 9 \geq -3\) which is TRUE.
        \item The expression reads as "There exists at least one value of x where P(x) does not hold." Looking at (b), we can conclude that if x is between 0 and 1, it will be FALSE and prove the expression.
        \item The expression reads as "There exists at least one value of x where P(x) holds." Looking at (a) and (c), we can conclude that if x is in the domain between \((-\infty,0) \cup (1, \infty)\), it will be TRUE and prove the expression.
        \item The expression reads as "All values of x make P(x) hold", which is not true from the problems (a)-(c) and my explanations in (d) and (e).
    \end{enumerate}


\clearpage
\section*{Question 2}
    \textbf{For each of the following logical statements, determine whether it is a tautology (always true), a contradiction (always false), or neither. For tautologies/contradictions, use logical equivalences to show it. If the statement is neither (a contingency), provide two counterexamples: true and false cases. Use logical equivalences to simplify the statement first. In both cases, state the names of the equivalences used in your proof or simplification. DO NOT use truth tables in this question.}
    \begin{enumerate}[label=(\alph*)]
        \item \((\neg(p\land q) \implies (\neg q \land p)) \lor (q \land \neg p)\)
        \item \(((\neg p \implies q) \land (q \lor (p \implies \neg q))) \lor (\neg q \implies \neg (p \land \neg q))\)
    \end{enumerate}
    \subsection*{Solution}
    \begin{enumerate}[label=(\alph*)]
        \item \((\neg(p\land q) \implies (\neg q \land p)) \lor (q \land \neg p)\) \\
        \((\neg p \lor \neg q \implies (\neg q \land p)) \lor (q \land \neg p)\) De Morgan's Law \\
        \((\neg (\neg p \lor \neg q) \lor (\neg q \land p)) \lor (q \land \neg p)\) Logical Equivalences \\
        \((p \land q \lor (\neg q \land p))  \lor (q \land \neg p)\) Double Negation and De Morgan's Law \\
        \((p \land (q \lor \neg q)) \lor (q \land \neg p)\) Distributive  and Absorption Laws \\
        \((p \land T) \lor (q \land \neg p)\) Negation Laws \\
        \(p \lor (q \land \neg p)\) Identity Laws \\
        \((p\lor q) \land (p \lor \neg p)\) Distributive Laws \\
        \((p \lor q) \land T\) Negation Laws \\
        \(p \lor q\) Identity Laws \\
        Contingency as it is neither F or T
        
        \item \(((\neg p \implies q) \land (q \lor (p \implies \neg q))) \lor (\neg q \implies \neg (p \land \neg q))\) \\
        \((\neg(\neg p \implies q) \land (q \lor (\neg(p \implies \neg q)))) \lor (\neg q \implies \neg (p \land \neg q))\) De Morgan's Laws \\
        \(( p \lor q \land (q \lor (\neg p \lor \neg q)) \lor (\neg q \implies \neg p \lor  q)\) De Morgan's and Double Negation Laws \\
        \((p \lor q \land ((q \lor \neg q) \lor \neg p)) \lor (q \lor \neg p)\) Distributive Laws and Grouping \\
        \((p \lor q \land (T \lor \neg p)) \lor (q \lor \neg p)\) Negation Laws \\
        \((p \lor q) \lor (q \lor \neg p)\) Domination and Identity Laws \\
        \((p \lor \neg p) \lor (q \lor q)\) Grouping \\
        \(T \lor q\) Negation and Idempotent Laws \\
        Tautology as it ends with T through Domination Laws
                
        
    \end{enumerate}

\clearpage
\section*{Question 3}
    \textbf{ Determine whether the following propositional statements are satisfiable. If a propositional statement is not satisfiable, use equivalences to show it is equivalent to F. (State the names of the equivalences used in your proof). If a propositional statement is satisfiable, provide an assignment of truth values to the variables that makes the statement true.}
    \begin{enumerate}[label=(\alph*)]
        \item \((p \land (q \lor \neg p)) \land \neg q\)
        \item \(((\neg p \lor q) \land \neg r) \lor (p \land \neg r)\)
        \item \((p \land (\neg q \lor r)) \implies (\neg r \land p)\)
    \end{enumerate}
    \subsection*{Solution}
    \begin{enumerate}[label=(\alph*)]
        \item \((p \land (q \lor \neg p)) \land \neg q\) \\
        \(((p \land q) \lor (p \land \neg p)) \land \neg q\) Distributive Laws \\
        \(((p \land q) \lor F) \land \neg q\) Negation Laws \\
        \((p \land q) \land \neg q\) Identity Laws \\
        \(p \land (q \land \neg q)\) Associative Laws \\
        \(p \land F\) Negation Laws \\
        Contradiction as it is F due to Domination Laws, NOT SATISFIABLE
        \item \(((\neg p \lor q) \land \neg r) \lor (p \land \neg r)\) \\
        \((\neg p \land \neg r ) \lor (q \land \neg r) \lor (p \land \neg r)\) Distributive Laws \\
        \((\neg r \land \neg p) \lor (\neg r \land p) \lor (\neg r \land q)\) Associative Laws \\
        \((\neg r \land (\neg p \lor p)) \lor (\neg r \land q)\) Distributive Laws \\
        \((\neg r \land T) \lor (\neg r \land q)\) Negation Laws \\
        \(\neg r \land (\neg r \land q)\) Identity Laws \\
        If p = T, q = T, and r = F, the expression is SATISFIABLE
        \item \((p \land (\neg q \lor r)) \implies (\neg r \land p)\) \\
        \(\neg (p \land (\neg q \lor r)) \lor (\neg r \land p)\) Logical Equivalences \\
        \(\neg (p \lor (q \land \neg r)) \lor (\neg r \land p)\) De Morgan's Laws \\
        \((\neg r \land p) \lor \neg p \lor (q \land \neg r)\) Associative Laws \\
        \(((\neg p \lor \neg r) \land (p \lor \neg p)) \lor (q \land \neg r)\) Distributive Laws and Grouping \\
        \(((\neg p \lor \neg r) \land T) \lor (q \land \neg r)\) Negation Laws \\
        \(\neg p \lor \neg r \lor (q \land \neg r)\) Identity Laws \\
        If p = F, q = T, and r = F, the expression is SATISFIABLE
        
    \end{enumerate}

\clearpage
\section*{Question 4}
    \textbf{Let the universe be all animals. Let \(R(x)\) be the predicate "x is a raccoon", \(D(x)\) be "x is a deer", \(V(x)\) be "x is violent", and \(F(x, y)\) be "x and y are friends". Express the following sentences using predicate logic.}\newline
    \textbf{Note: \(F(x, y)\) and \(F(y, x)\) are logically equivalent for any x and y. Both are acceptable.}
    \begin{enumerate}[label=(\alph*)]
        \item There exists a violent deer that is not friends with any raccoon.
        \item Every raccoon that is not violent is friends with at least one deer.
    \end{enumerate}

\clearpage
\section*{Question 5}
    \textbf{Let the universe be the set of all integers. Let \(S(x)\) represent "x is a square number" and \(N(x)\) represent "x is negative" We want to say that "every square number is non-negative". Which one of these is a correct way to state this using formal logic, and why are other options incorrect? Explain.}
    \begin{enumerate}[label=(\alph*)]
        \item \(\neg \exists (S(x) \implies N(x))\)
        \item \(\forall x(S(x) \implies \neg N(x))\)
        \item \(\forall x(S(x) \land \neg N(x))\)
        \item \(\exists x (S(x) \land \neg N(x))\)
    \end{enumerate}

\clearpage
\section*{Question 6}
    \textbf{Which of the two expressions correctly describes \(\exists !xP(x)\), that is, "there exists a unique x such that P(x) is true"? Explain your answer.}
    \begin{enumerate}[label=(\alph*)]
        \item \(\exists x (P(x) \land \forall y (P(y) \implies x = y))\)
        \item \(\exists x \forall y (P(y) \implies x =y)\)
    \end{enumerate}

\clearpage
\section*{Question 7}
    \textbf{Write the negation of \(\forall x\exists y\forall z(P(x,y) \implies Q(y) \lor R(x,z))\) such that the negation symbols immediately precede the predicates. Show your steps to receive full credit.}

\clearpage
\section*{Question 8}
    \textbf{Alice is a hardworking CS 182 student who has decided to achieve an A+ in the course. Consider the following premises:
    \begin{enumerate}
        \item If the alarm rings in the morning, Alice wakes up early.
        \item If Alice wakes up early, Alice studies in the morning.
        \item If Alice did not wake up early, Alice does not study in the morning.
        \item If the alarm did not ring, Alice does not wake up early.
    \end{enumerate}
    Using propositional logic and logical equivalence laws, prove or disprove the claim: The alarm rings if and only if Alice studies.}




\end{document}

