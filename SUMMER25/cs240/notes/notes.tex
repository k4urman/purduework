\documentclass{article}
\usepackage{amsmath}
\usepackage{listings}
\usepackage{xcolor}

\title{CS240 Notes}
\author{Maninder (Kaurman) Kaur}
\date{\today}

\begin{document}

\maketitle

\section*{Version 3}
\begin{lstlisting}[language=C]
/* version 3 of z = x * y
   reads the numbers to be subtracted from keyboard
   using the standard I/O library function scanf()
   and outputs the result on the terminal
   using printf() */

#include <stdio.h>

int main()
{
    int x;
    int y, z;

    // read input
    scanf("%d %d", &x, &y);

    /* compute multiplication */
    z = x * y;

    // print result
    printf("%d * %d = %d\n", x, y, z);
}
\end{lstlisting}

\texttt{main()} calls \texttt{scanf()} to do something for it; the two inputs that should be read from the user should be stored into \texttt{int x} and \texttt{y}. This is done by putting every function from \texttt{MAIN MEMORY}, where they get their own working area. It is allocated for the function to use, allowing \texttt{main()} to call and use \texttt{scanf()}. Passing functions means to use them.

Alice and Bob are friends. She writes him two letters, placing them in mailbox 5 and 7 at the UPS office. Bob comes in later and opens 5 and 7 for each letter. Alice and Bob represent the main function and the memory, while the letters are the functions.

Imagine memory as a bunch of slots that allow you to place data like bytes. Each slot allows 8 bits. The memory slots start at index 0 and go up to \( 2^{n}-1\) slots. Integers take up 4 bytes.

\subsection*{How is this different than printf?}
\texttt{main()} calls \texttt{printf()} to print on the terminal. It will print just the input of the variable. There is no need to store anything. \texttt{scanf} needs to know the address, while \texttt{printf} does not.

\subsection*{Segmentation Fault}
A segmentation fault occurs when you try to access a data value that the OS does not give access to.

\section*{Version 4}
\begin{lstlisting}[language=C]
/* 	version 4 of z = x * y
	same as version 3 but supports
	real numbers */

#include <stdio.h>

int main()
{
    float x, y, z;

    // read input
    scanf("%f %f", &x, &y);

    // multiply
    z = x * y;

    // print result
    printf("result of %f times %f is %f\n", x, y, z);
}
\end{lstlisting}

\section*{Version 5}
\begin{lstlisting}[language=C]
/* 	version 5 of z = x * y
	same as version 4 but uses separate
	function multiply2() to perform multiplication */

#include <stdio.h>

float multiply2(float, float);

void main()
{
    float x, y, z;

    // read input
    scanf("%f %f", &x, &y);

    // compute
    z = multiply2(x, y);

    // print result
    printf("result of %f * %f is %.3f\n", x, y, z);
}

/*	function multiply2(a,b) takes two
	arguments of type float, multiplies a
	and b, and returns the result to
	the calling function */

float multiply2(float a, float b)
{
    float c;

    // multiply a with b
    // and store the result in local variable c
    c = a * b;

    // return value of c to calling function
    return c;
}
\end{lstlisting}

\texttt{printf()} works as follows: if there is a variable \texttt{x} and we assign it a value, to print it we would simply use \texttt{printf(\%d, x)}. However, with \texttt{scanf}, we would use \texttt{scanf("\%d", \&x)}. We use \& because we are not passing the value of \texttt{x}, but using the memory address itself to store the value.

\end{document}
