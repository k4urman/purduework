\documentclass{article}
\usepackage{amsmath}
\usepackage{listings}
\usepackage{xcolor}

\title{CS240 Notes}
\author{Maninder (Kaurman) Kaur}
\date{\today}

\begin{document}

\maketitle

\section*{Week 1}
\subsection*{Version 3}
\begin{lstlisting}[language=C]
/* version 3 of z = x * y
   reads the numbers to be subtracted from keyboard
   using the standard I/O library function scanf()
   and outputs the result on the terminal
   using printf() */

#include <stdio.h>

int main()
{
    int x;
    int y, z;

    // read input
    scanf("%d %d", &x, &y);

    /* compute multiplication */
    z = x * y;

    // print result
    printf("%d * %d = %d\n", x, y, z);
}
\end{lstlisting}

\texttt{main()} calls \texttt{scanf()} to do something for it; the two inputs that should be read from the user should be stored into \texttt{int x} and \texttt{y}. This is done by putting every function from \texttt{MAIN MEMORY}, where they get their own working area. It is allocated for the function to use, allowing \texttt{main()} to call and use \texttt{scanf()}. Passing functions means to use them.

Alice and Bob are friends. She writes him two letters, placing them in mailbox 5 and 7 at the UPS office. Bob comes in later and opens 5 and 7 for each letter. Alice and Bob represent the main function and the memory, while the letters are the functions.

Imagine memory as a bunch of slots that allow you to place data like bytes. Each slot allows 8 bits. The memory slots start at index 0 and go up to \( 2^{n}-1\) slots. Integers take up 4 bytes.

\subsection*{How is this different than printf?}
\texttt{main()} calls \texttt{printf()} to print on the terminal. It will print just the input of the variable. There is no need to store anything. \texttt{scanf} needs to know the address, while \texttt{printf} does not.

\subsection*{Segmentation Fault}
A segmentation fault occurs when you try to access a data value that the OS does not give access to.

\subsection*{Version 4}
\begin{lstlisting}[language=C]
/* 	version 4 of z = x * y
	same as version 3 but supports
	real numbers */

#include <stdio.h>

int main()
{
    float x, y, z;

    // read input
    scanf("%f %f", &x, &y);

    // multiply
    z = x * y;

    // print result
    printf("result of %f times %f is %f\n", x, y, z);
}
\end{lstlisting}

\subsection*{Version 5}
\begin{lstlisting}[language=C]
/* 	version 5 of z = x * y
	same as version 4 but uses separate
	function multiply2() to perform multiplication */

#include <stdio.h>

float multiply2(float, float);

void main()
{
    float x, y, z;

    // read input
    scanf("%f %f", &x, &y);

    // compute
    z = multiply2(x, y);

    // print result
    printf("result of %f * %f is %.3f\n", x, y, z);
}

/*	function multiply2(a,b) takes two
	arguments of type float, multiplies a
	and b, and returns the result to
	the calling function */

float multiply2(float a, float b)
{
    float c;

    // multiply a with b
    // and store the result in local variable c
    c = a * b;

    // return value of c to calling function
    return c;
}
\end{lstlisting}

\texttt{printf()} works as follows: if there is a variable \texttt{x} and we assign it a value, to print it we would simply use \texttt{printf(\%d, x)}. However, with \texttt{scanf}, we would use \texttt{scanf("\%d", \&x)}. We use \& because we are not passing the value of \texttt{x}, but using the memory address itself to store the value.

\newpage
\section*{Week 2}
\subsection*{Linking and Loading}

    When we have a function like multiply2(). GCC will link the function statically, means that after being translated into machine code, it will be integrated into a.out. I called from these files printf() and scanf(), both pre-written code, which is an example of how often code for others and not just ourselfs. The code we make is deposited into a library to be used which is used in the process of Linking and Loading.

    We will typically dynamically link. Lets use the example that we use borg02, the server will share all the machine code from some library, which reduces the memory consumption. One copy of a function in a library in the usr/ directories.

    Loading is about loading a code segemnt to allow it to become an executable and loaded into main memory.

    \subsection*{Version 1}
    \begin{lstlisting}[language=C]
// Program to illustrate content vs. address
// of a local variable.

#include <stdio.h>

int main()
{
int x;

  x = 7;
  printf("%d\n",x);

  // format %p is for printing address
  printf("%p\n",&x);

}
    \end{lstlisting}

    When running a.out of Version 1, the ouput integer is printed and then the adress of where it is held. 0x indicates that the adress is hexidecimal which allows 4-bit systems. Each slot is given a memory adress. We use the numbers 0-9 and letters A-F. (i.e. Look at hexidecimal computer conversion).

    Each byte outputs 4 bits, and there are 12 bytes after 0x, indicating there will be 48-bits will be outputted in total.

    \subsection*{Version 2}
    \begin{lstlisting}[language=C]
// Meaning of a pointer: a variable whose content is an address.

#include <stdio.h>

int main()
{
int x, *y;

  x = 7;
  printf("%d\n", x);
  printf("%p\n", &x);

  y = &x;
  printf("%d\n", *y);
  printf("%p\n", y);

  // meaning of int **z?
}
    \end{lstlisting}

    \subsection*{Version 3}
    \begin{lstlisting}[language=C]
// Meaning of a pointer: a variable whose content is an address.

#include <stdio.h>

int main()
{
int x, *y;

  x = 7;
  printf("%d\n", x);
  printf("%p\n", &x);

  y = &x;
  printf("%d\n", *y);
  printf("%p\n", y);

  // meaning of int **z?
}
    \end{lstlisting}

    \subsection*{Version 4}
    \begin{lstlisting}[language=C]
    // Use functions changeling1() and changeling2() to illustrate
// passing by value vs. reference (i.e., address).

#include <stdio.h>

void changeling1(int);
void changeling2(int *);

int main()
{
int r;

  r = 7;
  changeling1(r);
  printf("%d\n", r);

  r = 9;
  changeling2(&r);
  printf("%d\n", r);

}


void changeling1(int x)
{
  x = 100;
}


void changeling2(int *y)
{
  *y = 200;
}
    \end{lstlisting}

    \subsection*{Version 7}

    \begin{lstlisting}[language=C]
// Pointers and arrays: 1-D array.

#include <stdio.h>

int main()
{
int z[5];

  z[0] = 100;
  z[1] = 200;
  z[2] = 300;
  z[3] = 400;
  z[4] = 500;

  printf("%d\n",z[0]);
  printf("%d\n",z[1]);
  printf("%d\n",z[2]);
  printf("%d\n",z[3]);
  printf("%d\n",z[4]);

  printf("%d\n",*z);
  printf("%d\n",*(z+1));
  printf("%d\n",*(z+2));
  printf("%d\n",*(z+3));
  printf("%d\n",*(z+4));

  *z = 1;
  *(z+1) = 2;
  *(z+2) = 3;
  *(z+3) = 4;
  *(z+4) = 5;

  printf("%d\n",z[0]);
  printf("%d\n",z[1]);
  printf("%d\n",z[2]);
  printf("%d\n",z[3]);
  printf("%d\n",z[4]);

}
    \end{lstlisting}

Unlike making serveral different int values a,b,c, we make a array called z[3]where all the values are adjacent in the memory block. If z[1] if at adress 100, the next int value z[1] will be at adress 104, because int values are of 4 bytes.

This 1-D array allows for random access which means that it makes the memory slots adjacent. The only memory adress we need to remember is the slot of index 0 (the starting).

To access the value of z[0], we should call *z. It will automatically start at the starting location. If we wanted to access z[2], we would use *(x+2) (or *(x+8) because 2 ints is 4*2 bytes).

\subsection*{Version 8}

\begin{lstlisting}[language=C]
// Pointers and arrays: valid vs. invalid memory.

#include <stdio.h>

int main()
{
int *z;

  *z = 100;
  *(z+1) = 200;
  *(z+2) = 300;

  printf("%d\n",*z);
  printf("%d\n",*(z+1));
  printf("%d\n",*(z+2));

}
\end{lstlisting}

Notice how we don't have a definitive idea of where *z is pointing at. Because we don't know where the adress is, it will probably end in a segmentation fault.

\subsection*{Version 9}

\begin{lstlisting}[language=C]
// Pointers, memory, and silent run-time errors.

#include <stdio.h>

int main()
{
int s[5];
int i;

  for(i=0; i<5; i++)
	s[i] = i;

  for(i=0; i<5; i++)
  	printf("%d\n",s[i]);

  // doing something sketchy
  for(i=0; i<6; i++)
	s[i] = i;

  for(i=0; i<6; i++)
	printf("%d\n",s[i]);

}
\end{lstlisting}

Notice how there are only 5 slots in s[]. However, we are asking to go past that threshold in the for loops. This does not cause a crash, but practices such as these should not be normalized. These are called "silent errors".
However, if we increase the max count from 6 to something higher, it will crash and give us a message saying "stack smashing detected; Terminated Abort".
This is called \textbf{Stack Overflow}, because the function attempts to access outside of the memory slots allocated for the array.

Memory slots hold a couple of things:
\begin{enumerate}
    \item Local Variables
    \item Arguments
    \item Return Addresses
\end{enumerate}

\end{document}
